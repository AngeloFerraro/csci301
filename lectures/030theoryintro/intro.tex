\documentclass{beamer}
\usetheme{Singapore}

%\usepackage{pstricks,pst-node,pst-tree}
\usepackage{amssymb,latexsym}
\usepackage{graphicx}
\usepackage{fancyvrb}
\usepackage{hyperref}

\newcommand{\bi}{\begin{itemize}}
\newcommand{\ii}{\item}
\newcommand{\ei}{\end{itemize}}
\newcommand{\Show}[1]{\psshadowbox{#1}}


\newcommand{\grf}[2]{\centerline{\includegraphics[width=#1\textwidth]{#2}}}
\newcommand{\tw}{\textwidth}
\newcommand{\bc}{\begin{columns}}
\newcommand{\ec}{\end{columns}}
\newcommand{\cc}[1]{\column{#1\textwidth}}

\newcommand{\bfr}[1]{\begin{frame}[fragile]\frametitle{{ #1 }}}
\newcommand{\efr}{\end{frame}}

\newcommand{\cola}[1]{\begin{columns}\begin{column}{#1\textwidth}}
\newcommand{\colb}[1]{\end{column}\begin{column}{#1\textwidth}}
\newcommand{\colc}{\end{column}\end{columns}}

\title{Introduction to Theory of Computation}
\author{Chapter 1}

\RecustomVerbatimEnvironment{Verbatim}{Verbatim}{frame=single}

\begin{document}
\begin{frame}
\maketitle

\end{frame}

\bfr{Purpose of the Theory of Computation}

Develop formal mathematical models of computation
that reflect real-world computers.

\bi
\ii What are the mathematical properties of computer hardware and
software? 
\ii What is a computation and what is an algorithm? Can we give rigorous
mathematical definitions of these notions?
\ii What are the limitations of computers? Can “everything” be computed?
\ei

\end{frame}

\bfr{Three areas of computer theory}
\bi
\ii Complexity Theory
\ii Computability Theory
\ii Automata Theory
\ei
\end{frame}

\bfr{Complexity Theory}
\bi
\ii Classify problems according to their degree of ``difficulty.''
Give a rigorous proof that problems that seem to be ``hard'' are
really ``hard.''
\ei
\end{frame}

\bfr{Computability Theory}
\bi
\ii In 1930's G\"odel, Turing and Church discovered that some
fundamental mathematical problems cannot be solved by a computer.
\ii Classify problems as being solvable or unsolvable.
\ei
\end{frame}

\bfr{Automata Theory}
\bi
\ii Three models:
\bi
\ii Finite Automata.  These are used in text processing, compilers, and
hardware design.
\ii Context-Free Grammars. These are used to define programming lan-
guages and in Artificial Intelligence.
\ii Turing Machines. These form a simple abstract model of a “real”
computer, such as your PC at home.
\ei
\ii Do these models have the same power, or can one model solve more
problems than the other?
\ei
\end{frame}


\bfr{This course}
\bi
\ii Start with Automata Theory, followed by Computability Theory.
\ii Complexity is covered in algorithms courses.
\ei
\end{frame}
\bfr{Practical applications}

\begin{enumerate}
\ii It is about mathematical properties of computer hardware and software.
\ii This theory is very much relevant to practice, for example, in the design
of new programming languages, compilers, string searching, pattern
matching, computer security, artificial intelligence, {\em etc.}
\ii This course helps you to learn problem solving skills. Theory teaches
you how to think, prove, argue, solve problems, express, and abstract.
\ii This theory simplifies the complex computers to an abstract and simple
mathematical model, and helps you to understand them better.
\ii This course is about rigorously analyzing capabilities and limitations
of systems.
\end{enumerate}

\end{frame}



\end{document}