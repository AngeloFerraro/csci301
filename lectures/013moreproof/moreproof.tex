\documentclass{beamer}
\usetheme{Singapore}

\usepackage{amsmath,amssymb,latexsym}
\usepackage{graphicx}
\usepackage{fancyvrb}
\usepackage{hyperref}

\newcommand{\bi}{\begin{itemize}}
\newcommand{\ii}{\item}
\newcommand{\ei}{\end{itemize}}
\newcommand{\Show}[1]{\psshadowbox{#1}}

\newcommand{\set}[1]{\ensuremath{\left\{ #1 \right\}}}
\newcommand{\nats}{\ensuremath{\mathbb{N}}}
\newcommand{\nni}{\ensuremath{\mathbb{N}^0}}
\newcommand{\ints}{\ensuremath{\mathbb{Z}}}
\newcommand{\power}{\ensuremath{\mathcal{P}}}
\renewcommand{\neg}{\sim}
\newcommand{\xor}{\oplus}
\newcommand{\then}{\ensuremath{\Rightarrow}}
\newcommand{\lcm}{\mbox{lcm}}
\newcommand{\QED}{\hfill\ensuremath{\blacksquare}}

\newcommand{\grf}[2]{\centerline{\includegraphics[width=#1\textwidth]{#2}}}
\newcommand{\tw}{\textwidth}
\newcommand{\bc}{\begin{columns}}
\newcommand{\ec}{\end{columns}}
\newcommand{\cc}[1]{\column{#1\textwidth}}

\newcommand{\bfr}[1]{\begin{frame}[fragile]\frametitle{{ #1 }}}
\newcommand{\efr}{\end{frame}}

\newcommand{\cola}[1]{\begin{columns}\begin{column}{#1\textwidth}}
\newcommand{\colb}[1]{\end{column}\begin{column}{#1\textwidth}}
\newcommand{\colc}{\end{column}\end{columns}}

\title{Book of Proof: Part III, More on Proof}

\RecustomVerbatimEnvironment{Verbatim}{Verbatim}{frame=single}

\begin{document}
\begin{frame}
\maketitle

\end{frame}

\bfr{If-and-Only-If Proof}

\begin{center}

  {\bf Outline for If-and-Only-If Proof}
  \fbox{\parbox{0.8\textwidth}{
      {\bf Proposition} $P$ if and only if $Q$.

      {\it Proof.}
\\\\
      ``Only if''
      
      [Prove $P\then Q$ by whatever means you can.]
\\\\
      ``If''
      
      [Prove $Q\then P$ by whatever means you can.]
    }
  }
\end{center}

\end{frame}

\bfr{Equivalent Statements}

{\bf Theorem} Suppose $A$ is an $n\times n$ matrix.  The following
statements are equivalent:
\renewcommand{\theenumi}{\alph{enumi}}
\begin{enumerate}
\item $A$ is invertible.
\item $Ax=b$ has a unique solution for every $b\in\mathbb{R}^n$.
\item $Ax=0$ has only the trivial solution.
\item The reduced row echelong form of $A$ is $I_n$.
\item $\det(A)\neq 0$.
\item The matrix $A$ does not have 0 as an eigenvector.
\end{enumerate}

\end{frame}

\bfr{Equivalent Statements}
\[
\begin{array}{ccccc}
  a & \Rightarrow & b & \Rightarrow & c\\
  \Uparrow &&&&\Downarrow \\
  f & \Leftarrow & e & \Leftarrow &  d
\end{array}
\]
\pause\vfill
\[
\begin{array}{ccccc}
  a & \Rightarrow & b & \Leftrightarrow & c\\
  \Uparrow &&\Downarrow&&\\
  f & \Leftarrow & e & \Leftrightarrow &  d
\end{array}
\]
\pause\vfill
\[
\begin{array}{ccccc}
  a & \Leftrightarrow & b & \Leftrightarrow & c\\
   &&\Updownarrow  \\
  f & \Leftrightarrow & e & \Leftrightarrow &  d
\end{array}
\]


\end{frame}

\bfr{Existence Proofs}

{\bf Proposition} There exists an even prime number.
\pause

{\it Proof.} Two is an even prime number.

\pause\vfill

{\bf Proposition}  There exists an integer that can be expressed as
the sum of two perfect cubes in two different ways.
\pause

{\it Proof.}
\begin{align*}
  1^3 + 12^3 &= 1729\\
  9^3 + 10^3 &= 1729
\end{align*}
\end{frame}


\bfr{Example}

{\bf Proposition 7.1} If $a,b\in\nats$ then \\there exist
$k,\ell\in\ints$ for which $\gcd(a,b)=ak+b\ell$.

\pause

{\it Proof.}  Suppose $a,b\in\nats$.

Consider the set $A=\set{ax+by : x,y\in\ints}$.

$A$ contains positive integers and 0.

Let $d\in A$ be the smallest positive integer.

$d = ak + b\ell$ for some $k,\ell\in\ints$.

We will show that $d=\gcd(a,b)$.

First, prove that $d\mid a$ and $d\mid b$.

Then show that it is the largest such number.


\end{frame}
\end{document}
