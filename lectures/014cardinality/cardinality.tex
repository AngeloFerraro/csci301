\documentclass{beamer}
\usetheme{Singapore}

\usepackage{amsmath,amssymb,latexsym}
\usepackage{graphicx}
\usepackage{fancyvrb}
\usepackage{hyperref}

\newcommand{\bi}{\begin{itemize}}
\newcommand{\ii}{\item}
\newcommand{\ei}{\end{itemize}}
\newcommand{\Show}[1]{\psshadowbox{#1}}

\newcommand{\set}[1]{\ensuremath{\left\{ #1 \right\}}}
\newcommand{\nats}{\ensuremath{\mathbb{N}}}
\newcommand{\nni}{\ensuremath{\mathbb{N}^0}}
\newcommand{\ints}{\ensuremath{\mathbb{Z}}}
\newcommand{\power}{\ensuremath{\mathcal{P}}}
\renewcommand{\neg}{\sim}
\newcommand{\xor}{\oplus}
\newcommand{\then}{\ensuremath{\Rightarrow}}
\newcommand{\lcm}{\mbox{lcm}}
\newcommand{\QED}{\hfill\ensuremath{\blacksquare}}
\newcommand{\equivmod}[3]{#1 \equiv #2\ (\mbox{mod } #3)}
\newcommand{\nequivmod}[3]{#1 \not\equiv #2\ (\mbox{mod } #3)}

\newcommand{\grf}[2]{\centerline{\includegraphics[width=#1\textwidth]{#2}}}
\newcommand{\tw}{\textwidth}
\newcommand{\bc}{\begin{columns}}
\newcommand{\ec}{\end{columns}}
\newcommand{\cc}[1]{\column{#1\textwidth}}

\newcommand{\bfr}[1]{\begin{frame}[fragile]\frametitle{{ #1 }}}
\newcommand{\efr}{\end{frame}}

\newcommand{\cola}[1]{\begin{columns}\begin{column}{#1\textwidth}}
\newcommand{\colb}[1]{\end{column}\begin{column}{#1\textwidth}}
\newcommand{\colc}{\end{column}\end{columns}}

\title{Book of Proof: Part IV, Relations, Functions, and Cardinality}

\RecustomVerbatimEnvironment{Verbatim}{Verbatim}{frame=single}

\begin{document}
\begin{frame}
\maketitle

\end{frame}

\bfr{Relations}
\[
\begin{array}{ccc}
  5 < 10 & 3 < 12 & 99 < 999 \\\\
  5 \not< 5 & 12\not< 3 & 10 \not< 0
\end{array}
\]

\pause\vfill
\[ R = \set{(5,10), (3,12), (99,999), \ldots} \]

\[
\begin{array}{ccc}
  (5,10)\in R & (3,12)\in R & (99,999)\in R \\\\
  (5,5) \not\in R & (12,3) \not\in R & (10,0)\not\in R
\end{array}
\]


\end{frame}

\bfr{Relations}

{\bf Definition 11.1}  A {\bf relation} on a set $A$ is a subset
$R\subseteq A\times A$.

We abbreviate $(x,y)\in R$ as $xRy$.

\end{frame}


\bfr{Relations in Pictures}

Let $B=\set{0,1,2,3,4,5}$ and
\[ U = \set{
  (1,3), (3,3), (5,2), (2,5), (4,2)
} \subseteq B\times B
\]

\grf{0.5}{relationUonB.png}

\end{frame}

\bfr{Properties of Relations}

{\bf Definition 11.2}  Suppose $R$ is a relation on set $A$.

\begin{enumerate}
\item  $R$ is {\bf reflexive} if $xRy$ for every $x\in A$.
  \[ \forall x\in A, xRx \]

\item $R$ is {\bf symmetric} if $xRy$ implies $yRx$ for all
  $x,y\in A$.
  \[ \forall x,y \in A, xRy \then yRx \]

\item $R$ is {\bf transitive} if $xRy$ and $yRz$ imply $xRz$.
  \[
  \forall x,y,z\in A, ((xRy) \land (yRz)) \then xRz
  \]
\end{enumerate}
\end{frame}

\bfr{Pictures of Relation Properties}

\grf{1.0}{relationpics.png}

\end{frame}


\bfr{Relations on \ints}

\grf{1.0}{relationsonZ.png}


\end{frame}

\bfr{Equivalence relations}

{\bf Definition 11.3}  A relation $R$ on a set $A$ is an {\bf
  equivalence relation} if it is symmetric, reflexive, and
transitive.

\pause\vfill

{\bf Definition 11.4} Suppose $R$ is an equivalence relation on set
$A$. Given any element $a\in A$, the {\bf equivalence class containing
  $a$} is the subset $\set{x\in A : xRa}$ of $A$ consisting of all
elements of $A$ that relate to $a$.

This set is denoted $[a]$:
\[
  [a] = \set{x \in A : xRa}
  \]
  
\end{frame}


\bfr{Pictures of equivalence relations}

\grf{1.0}{equivalencepics.png}

\end{frame}

\bfr{Congruence as equivalence relations}

Example 11.8 proved that $\equivmod{}{}{n}$ is an equivalence relation.


\begin{align*}
xRy &= \set{(x,y) : \equivmod{x}{y}{3}} \\\\
[0] &= \set{x\in\ints : \equivmod{x}{0}{3}} \\
&=  \set{x\in\ints : 3\mid (x-0)}
=  \set{x\in\ints : 3\mid x}\\
&= \set{...,-6,-3,0,3,6,9,...} = [3] = [6]\\
[1] &= \set{x\in\ints : \equivmod{x}{1}{3}} \\
&=  \set{x\in\ints : 3\mid (x-1)}\\
&= \set{...,-5,-2,1,4,7,10,...} = [4] = [7]\\
[2] &= \set{x\in\ints : \equivmod{x}{2}{3}} \\
&=  \set{x\in\ints : 3\mid (x-2)}\\
&= \set{...,-4,-1,2,5,8,11,...} = [5] = [7]\\
\end{align*}

\end{frame}

\bfr{Partitions}

{\bf Definition 11.5}  A {\bf partition} of a set $A$ is a set of
non-empty subsets of $A$, such that the union of all the subsets
equals $A$, and the intersection of any two different subsets is
$\emptyset$.

\vfill

$\set{[0],[1],[2]}$ under the relation $\equivmod{}{}{3}$, is a
partition of $\ints$:
\begin{align*}
\set{[0],[1],[2]} &= \{\set{...,0,3,6,...},
\set{...,1,4,7,...},
\set{...,2,5,8,...}\}
\end{align*}

\end{frame}

\bfr{Equivalence Relations and Partitions}

{\bf Theorem 11.2}  Suppose $R$ is an equivalence relation on set
$A$.  The the set $\set{[a] : a\in A}$ of equivalence classes of $R$
forms a partition of $A$.


\vfill

Conversely, any parition of $A$ describes an equivalence relation $R$
where $xRy$ if and only if $x$ and $y$ belong to the same set in the
parition.
\end{frame}


\bfr{The Integers Modulo $n$}

\grf{1.0}{intsmodfive.png}

\[ \ints_5 = \set{[0],[1],[2],[3],[4]} \]

\end{frame}

\bfr{Relations Between Sets}

{\bf Definition 11.7}  A {\bf relation} from a set $A$ to a set $B$ is
a subset $R\subseteq A\times B$.

We abbreviate the statement $(x,y)\in R$ as $xRy$.

\grf{0.5}{relationbetweenAandB.png}

\end{frame}

\bfr{Functions}

\end{frame}


\end{document}
