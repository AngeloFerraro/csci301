\documentclass{beamer}
%\usetheme{Singapore}
\usepackage{tikz}
\usetikzlibrary{arrows,automata}

%\usepackage{pstricks,pst-node,pst-tree}
\usepackage{amssymb,latexsym}
\usepackage{graphicx}
\usepackage{fancyvrb}
\usepackage{hyperref}

\newcommand{\bi}{\begin{itemize}}
\newcommand{\ii}{\item}
\newcommand{\ei}{\end{itemize}}
\newcommand{\Show}[1]{\psshadowbox{#1}}


\newcommand{\grf}[2]{\centerline{\includegraphics[width=#1\textwidth]{#2}}}
\newcommand{\tw}{\textwidth}
\newcommand{\bc}{\begin{columns}}
\newcommand{\ec}{\end{columns}}
\newcommand{\cc}[1]{\column{#1\textwidth}}

\newcommand{\bfr}[1]{\begin{frame}[fragile]\frametitle{{ #1 }}}
\newcommand{\efr}{\end{frame}}

\newcommand{\cola}[1]{\begin{columns}\begin{column}{#1\textwidth}}
\newcommand{\colb}[1]{\end{column}\begin{column}{#1\textwidth}}
\newcommand{\colc}{\end{column}\end{columns}}

\title{Recursion Theorem Notes}
\author{Geoffrey Matthews}

\RecustomVerbatimEnvironment{Verbatim}{Verbatim}{frame=single}

\newcommand{\pth}[5]{\path (#1) edge node {$\frac{#2}{#3 #4}$} (#5); }
\newcommand{\pthl}[5]{\path (#1) edge [bend left] node {$\frac{#2}{#3 #4}$} (#5); }
\newcommand{\pthr}[5]{\path (#1) edge [bend right] node {$\frac{#2}{#3 #4}$} (#5); }
\newcommand{\pthla}[5]{\path (#1) edge [loop above] node {$\frac{#2}{#3 #4}$} (#5); }
\newcommand{\pthlb}[5]{\path (#1) edge [loop below] node {$\frac{#2}{#3 #4}$} (#5); }
\newcommand{\pthlr}[5]{\path (#1) edge [loop right] node {$\frac{#2}{#3 #4}$} (#5); }
\newcommand{\pthll}[5]{\path (#1) edge [loop left] node {$\frac{#2}{#3 #4}$} (#5); }


\begin{document}
\begin{frame}
\maketitle

\end{frame}

\bfr{Self Reference}
\begin{itemize}
\item This sentence is false.\pause
\item Print the following twice, the second time in quotations:\\
  ``Print the following twice, the second time in quotations:''\pause
  \item ``yields falsehood when appended to its own quotation.''\\
  yields falsehood when appended to its own quotation.''
\end{itemize}

\end{frame}

\bfr{Quine Pages}
\begin{itemize}
\item  \url{https://www.nyx.net/~gthompso/quine.htm}
\item \url{http://www.madore.org/~david/computers/quine.html}
\end{itemize}
\end{frame}

\bfr{A Quine Turing Machine}
\begin{itemize}
\item Consider the TM $B$:
  \fbox{\parbox{3in}{
    $B$: on input $\langle M \rangle$:
    \begin{itemize}
    \item Create TM $A$:
      \fbox{\parbox{2in}{
        $A$: on input $w$:
        \begin{itemize}
        \item Erase $w$
        \item Write $\langle M \rangle$ on the tape.
        \item Simulate what's on the tape.
        \end{itemize}
      }}
  \item Write $\langle A\rangle$ on the tape.
  \end{itemize}
  }}
\item Now consider TM $A$:
      \fbox{\parbox{3in}{
        $A$: on input $w$:
        \begin{itemize}
        \item Erase $w$
        \item Write $\langle B \rangle$ on the tape.
        \item Simulate what's on the tape.
        \end{itemize}
      }}
  
\pause
\item Note that $B$ could do other things before writing $A$ and halting.
  \end{itemize}
\end{frame}

\bfr{Recursion theorem makes undecidability easier to prove}
\[
Halt = \{ \langle M,w\rangle : \mbox{$M$ is a TM that terminates on $w$}\}
\]

Assume there is a TM $H$ that decides this language.

Construct the following TM $Q$:

\centerline{\fbox{\parbox{3in}{
    $Q$ : On input $w$:
    \begin{itemize}
      \item Obtain description of self, $\langle Q \rangle$.
      \item Run $H$ on $\langle Q, w\rangle$.
      \item If $H$ accepts, loop forever, else halt.
    \end{itemize}
}}}

\end{frame}



\end{document}

