\documentclass[12pt]{article}
\usepackage{pstricks,pst-node}
\pagestyle{empty}
\setlength{\oddsidemargin}{0in}
\setlength{\evensidemargin}{0in}
\setlength{\textwidth}{6.5in}
\setlength{\textheight}{9in}
\begin{document}
\centerline{\large Proof the Busy Beaver function is uncomputable}

\begin{itemize}
\item
Let $bb(n)$ be the largest (finite) number of 1's output by a Turing Machine
with $n$ states.
\item
Suppose there is a Turing Machine $M_{bb}$ that computes $bb(n)$, that is,
starting with $n$ on the tape, the machine halts with $bb(n)$ on the tape.

\psset{arrows=->,arrowscale=2,colsep=1.5}
\begin{psmatrix}
\rnode{N}{
\begin{tabular}{ccc}\hline
 & 111 & \\\hline
 & $N$ & \\
\end{tabular}
}
&
\rnode{bbN}{
\begin{tabular}{ccc}\hline
 & 11111111 & \\\hline
 & $bb(N)$ & \\
\end{tabular}
}

\nccurve[angleA=-30,angleB=-150]{N}{bbN} \ncput*{$M_{bb}$}

\end{psmatrix}

\vspace{1cm}

\item
Let $g(n) = bb(2n)$.  We can build a TM for $g$ by starting with
a machine that doubles the input, and then runs the machine  $M_{bb}$.

\begin{psmatrix}
\rnode{N}{
\begin{tabular}{ccc}\hline
 & 111 & \\\hline
 & $N$ & \\
\end{tabular}
}
&
\rnode{NN}{
\begin{tabular}{ccc}\hline
 & 111111 & \\\hline
 & $2N$ & \\
\end{tabular}
}
&
\rnode{bbN}{
\begin{tabular}{ccc}\hline
 & 11111111111 & \\\hline
 & $bb(2N)$ & \\
\end{tabular}
}

\nccurve[angleA=-30,angleB=-150]{N}{NN} \ncput*{$M_{2}$}
\nccurve[angleA=-30,angleB=-150]{NN}{bbN} \ncput*{$M_{bb}$}

\end{psmatrix}

\vspace{1cm}


\item
Suppose the machine for $g$, $M_g$, which is a combination of $M_2$
and $M_{bb}$,  has $k$ states.


\item
For any natural number $n$,
we could build a new busy beaver machine that starts by 
putting $n$ 1's on the tape, and then runs the $M_g$ machine.



\begin{psmatrix}
\rnode{blank}{
\begin{tabular}{ccc}\hline
 & ~~~ & \\\hline
 & $\Lambda$ & \\
\end{tabular}
}
&
\rnode{N}{
\begin{tabular}{ccc}\hline
 & 111 & \\\hline
 & $N$ & \\
\end{tabular}
}
&
\rnode{NN}{
\begin{tabular}{ccc}\hline
 & 111111 & \\\hline
 & $2N$ & \\
\end{tabular}
}
&
\rnode{bbN}{
\begin{tabular}{ccc}\hline
 & 11111111111 & \\\hline
 & $bb(2N)$ & \\
\end{tabular}
}

\nccurve[angleA=-30,angleB=-150]{blank}{N} \ncput*{$M_{n}$}
\nccurve[angleA=-30,angleB=-150]{N}{NN} \ncput*{$M_{2}$}
\nccurve[angleA=-30,angleB=-150]{NN}{bbN} \ncput*{$M_{bb}$}

\end{psmatrix}

\vspace{1cm}

\item Now note:
\begin{itemize}
\item We can build this busy beaver with $n+k$ states.
\item The output of this busy beaver is $g(n) = bb(2n)$ 1's.
\end{itemize}

\item Now pick $n=2k$.  Then we can build a machine with $n+k = 3k$
states and its output is $bb(2(2k)) = bb(4k)$ 1's.

\item Therefore, if $M_{bb}$ exists, 
we could build a machine with $3k$ states that output
as many 1's as you can possibly do with $4k$ states. 

See the problem? 

\end{itemize}

\end{document}


