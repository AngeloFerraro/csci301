\documentclass{beamer}
\usepackage{fancyvrb}
\usepackage{hyperref}
\usepackage{multicol}
\usepackage{graphicx}
\newtheorem{theo}{Theorem}[section]

\newcommand{\myfig}[1]{\centerline{\includegraphics[scale=0.25]{figures/#1.png}}}

\newcommand{\trans}[5]{
\begin{tabular}{|c|c|c|c|c|}\hline
#1 & #2 & #3 & #4 & #5 \\\hline
\end{tabular}
}

\newcommand{\arr}{&\rightarrow&}
\newcommand{\darr}{&\Rightarrow&}
\newcommand{\ar}{\rightarrow}
\newcommand{\dar}{\Rightarrow}
\newcommand{\bee}{\begin{eqnarray*}}
\newcommand{\eee}{\end{eqnarray*}}
\newcommand{\lmb}{\ensuremath{\lambda}}

\newcommand{\bi}{\begin{itemize}}
\newcommand{\ii}{\item}
\newcommand{\ei}{\end{itemize}}

\newcommand{\sect}[1]{
\section{#1}
\begin{frame}[fragile]\frametitle{#1}
}

\mode<presentation>
{
%  \usetheme{Madrid}
  % or ...

%  \setbeamercovered{transparent}
  % or whatever (possibly just delete it)
}

\usepackage[english]{babel}

\usepackage[latin1]{inputenc}

\title
{
SICP Notes 03
}

\subtitle{
\url{https://mitpress.mit.edu/sicp/}\\
\url{http://www.aduni.org/courses/sicp/index.php}
} % (optional)

\author[Geoffrey Matthews]
{Geoffrey Matthews}
% - Use the \inst{?} command only if the authors have different
%   affiliation.

\institute[WWU/CS]
{
  Department of Computer Science\\
  Western Washington University
}
% - Use the \inst command only if there are several affiliations.
% - Keep it simple, no one is interested in your street address.

\date{\today}

% If you have a file called "university-logo-filename.xxx", where xxx
% is a graphic format that can be processed by latex or pdflatex,
% resp., then you can add a logo as follows:

%\pgfdeclareimage[height=0.5cm]{university-logo}{WWULogoProColor}
%\logo{\pgfuseimage{university-logo}}

% If you wish to uncover everything in a step-wise fashion, uncomment
% the following command: 

%\beamerdefaultoverlayspecification{<+->}

\begin{document}

\begin{frame}
  \titlepage
\end{frame}


\newcommand{\myref}[1]{\small\item\url{#1}}
\newcommand{\myreft}[1]{\footnotesize\item\url{#1}}

%\begin{frame}
%  \frametitle{Outline}
%  \tableofcontents
%  % You might wish to add the option [pausesections]
%\end{frame}


\sect{Higher Order Procedures}
\bi
\ii
Higher-order procedures are procedures that manipulate other procedures.
\ii
Procedures have first-class status in Scheme.
\bi
\ii
They may be named by variables.
\ii
They may be passed as arguments to procedures.
\ii
They may be returned as the results of procedures.
\ii
They may be included in data structures.
\ei
\ei
\end{frame}
\sect{Summing}
\[\sum_{i=1}^{5} i^2 = 1^2 + 2^2 + 3^2 + 4^2 + 5^2\]\pause
\bi
\ii If we were to write \verb|sum| as a procedure, what are the parameters?
\ei\pause
\begin{Verbatim}
(define sum
   (lambda (func a b)
      (if (> a b)
          0
          (+ (func a)
             (sum func
                  (+ 1 a)
                  b)))))
\end{Verbatim}


\end{frame}
\sect{Sum examples}
How do we compute $\sum_{i=1}^{n}i$?\pause
\begin{Verbatim}
(sum (lambda (i) i)
     1
     n)
\end{Verbatim}
\pause
How do we compute $\sum_{i=1}^{n}\frac{1}{i}$?\pause
\begin{Verbatim}
(sum (lambda (i) (/ 1 i))
     1
     n)
\end{Verbatim}
\pause
How about $\sum_{i=1}^{n}i^3+2i$?\pause
\begin{Verbatim}
(sum (lambda (x) (+ (* x x x) (* 2 x)))
     1
     n)
\end{Verbatim}
\end{frame}

\sect{Procedures that return procedures}

Here is a procedure for exponentiation:
\begin{Verbatim}
(define expt
   (lambda (b n)
      (if (= n 0)
          1
          (* b (expt b (- n 1))))))
\end{Verbatim}
How could we use this procedure to allow us to create procedures to
calculate $n^2$, $n^3$, $n^4$, {\em etc.}? 
\end{frame}
\sect{Procedures that return procedures}
\begin{Verbatim}
(define make-expt
   (lambda (n)
      (lambda (b) (expt b n))))
\end{Verbatim}
Now, let's use the make-expt procedure we just wrote to define square.
\begin{Verbatim}
(define square
   (make-expt 2))
\end{Verbatim}
And use make-expt to write cube:
\begin{Verbatim}
(define cube
   (make-expt 3))
\end{Verbatim}
How do we define a procedure to return $n^8$?
\begin{Verbatim}
(define power-8
   (make-expt 8))
\end{Verbatim}
\end{frame}


\sect{Multiplying: A Different Take on Sum}
\[
\prod_{i=1}^{5}i^2 = (1^2)(2^2)(3^2)(4^2)(5^2)
\]
\begin{Verbatim}
(define prod
   (lambda (term a b)
      (if (> a b)
          1
          (* (term a)
             (prod term
                   (+ 1 a)
                   b)))))
\end{Verbatim}
\end{frame}
\sect{Prod examples}
Use prod to calculate $\prod_{i=1}^{n}\frac{1}{i}$\pause
\begin{Verbatim}
(prod (lambda (x) (/ 1 x))
     1
     10)
\end{Verbatim}
\pause
How about $\prod_{i=2}^{100}i^2$\pause
\begin{Verbatim}
(prod (lambda (x) (* x x))
     2
     100)
\end{Verbatim}
\end{frame}

\sect{Let there be local variables}

The general format of the let statement is
\begin{Verbatim}
(let ((<var1> <exp1>)
      (<var2> <exp2>)
      ...
      (<varn> <expn>))
  <body>)
\end{Verbatim}
The let statement is equivalent to the following
\begin{Verbatim}
((lambda (<var1> <var2> ... <varn>)
     <body>)
 <exp1>
 <exp2>
 ... 
 <expn>)
\end{Verbatim}
\end{frame}
\sect{Let example}
For example,
\begin{Verbatim}
(let ((a 2)
      (b 3) 
      (c 4))
   (+ a b c))
\end{Verbatim}
de-sugars into
\begin{Verbatim}
((lambda (a b c) (+ a b c)) 2 3 4)
\end{Verbatim}
which would return 9.
\end{frame}
\sect{Let examples}
\begin{Verbatim}
(define procedure-with-let
   (lambda (x)
      (let ((a (+ x 2))
            (b (* x 3)))
         (+ a b))))
\end{Verbatim}
What will be returned by the following call?
\begin{Verbatim}
(procedure-with-let 4)
\end{Verbatim}
\pause
18

\end{frame}
\end{document}