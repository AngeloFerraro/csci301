\documentclass{beamer}
\usetheme{Singapore}

\usepackage{amsmath,amssymb,latexsym}
\usepackage{graphicx}
\usepackage{fancyvrb}
\usepackage{hyperref}

\newcommand{\bi}{\begin{itemize}}
\newcommand{\ii}{\item}
\newcommand{\ei}{\end{itemize}}
\newcommand{\Show}[1]{\psshadowbox{#1}}

\newcommand{\set}[1]{\ensuremath{\left\{ #1 \right\}}}
\newcommand{\nats}{\ensuremath{\mathbb{N}}}
\newcommand{\nni}{\ensuremath{\mathbb{N}^0}}
\newcommand{\ints}{\ensuremath{\mathbb{Z}}}
\newcommand{\power}{\ensuremath{\mathcal{P}}}
\renewcommand{\neg}{\sim}
\newcommand{\xor}{\oplus}
\newcommand{\then}{\ensuremath{\Rightarrow}}
\newcommand{\lcm}{\mbox{lcm}}
\newcommand{\QED}{\hfill\ensuremath{\blacksquare}}
\newcommand{\equivmod}[3]{#1 \equiv #2\ (\mbox{mod } #3)}
\newcommand{\nequivmod}[3]{#1 \not\equiv #2\ (\mbox{mod } #3)}

\newcommand{\grf}[2]{\centerline{\includegraphics[width=#1\textwidth]{#2}}}
\newcommand{\tw}{\textwidth}
\newcommand{\bc}{\begin{columns}}
\newcommand{\ec}{\end{columns}}
\newcommand{\cc}[1]{\column{#1\textwidth}}

\newcommand{\bfr}[1]{\begin{frame}[fragile]\frametitle{{ #1 }}}
\newcommand{\efr}{\end{frame}}

\newcommand{\cola}[1]{\begin{columns}\begin{column}{#1\textwidth}}
\newcommand{\colb}[1]{\end{column}\begin{column}{#1\textwidth}}
\newcommand{\colc}{\end{column}\end{columns}}

\title{Book of Proof: Part II, Conditionals}

\RecustomVerbatimEnvironment{Verbatim}{Verbatim}{frame=single}

\begin{document}
\begin{frame}
\maketitle

\end{frame}

\bfr{Proofs}

\begin{description}

\item[Theorem]  Something important you want to prove.
\item[Proposition]   Something not so important you want to prove.
\item[Corollary]  Something you want to prove in order to prove something else.

\end{description}
\end{frame}

\bfr{Definitions}

\begin{description}
\item[Definition 4.1] An integer $n$ is {\bf even} if $n=2a$ for some
  integer $z\in\ints$.
\item[Definition 4.1] An integer $n$ is {\bf odd} if $n=2a+1$ for some
  integer $z\in\ints$.
\item[Definition 4.3] Two integers have the {\bf same parity} if they
  are both even or they are both odd.  Otherwise they have {\bf
    opposite parity}.
\item[Definition 4.4] Suppose $a$ and $b$ are integers.  We say that
  $a$ {\bf divides} $b$, written $a\mid b$, if $b=ac$ for some
  $c\in\ints$.  In this case we also say that $a$ is a {\bf divisor}
  of $b$, and that $b$ is a {\bf multiple} of $a$.
\item[Definition 4.5] A natural number $n$ is {\bf prime} if it has
  exactly two positive divisors, 1 and $n$.

\end{description}
\end{frame}


\bfr{Definitions}
\begin{description}
  \item[Definition 4.6] The {\bf greatest common divisor} of integers
    $a$ and $b$, denoted $\gcd(a,b)$, is th largest integer that
    divides both $a$ and $b$.  The {\bf least common multiple} of
    non-zero integers $a$ and $b$, denoted $\lcm(a,b)$, is the
    smallest positive integer that is a multiple of both $a$ and $b$.
\end{description}

\begin{align*}
  \gcd(18,24) &= 6 & \gcd(5,5) &= 5\\
  \gcd(32,-8) &= 8 & \gcd(50,18) &= 2\\
  \gcd(50,9) &= 1 & \gcd(0,6) &= 6\\
  \gcd(0,0) &= \mbox{undefined}\\
  \lcm(4,6) &= 12 & \lcm(7,7) &=7
\end{align*}

\end{frame}

\bfr{Some facts accepted without proof}

If $a,b\in\ints$, then
 $ a+b\in\ints$,
$  a-b\in\ints$,
$  ab\in\ints$.

\vfill

\begin{description}
  \item[The Division Algorithm] Given integers $a$ and $b$ with $b>0$,
    there exist unique integers $q$ and $r$ for which $a=qb+r$ and
    $0\leq r < b$.
\end{description}

\vfill
Every natural number greater than 1 has a unique factorization into
primes. 

\end{frame}


\bfr{Direct Proof}
\begin{center}

If $P$, then $Q$.
\vfill

\begin{tabular}{|c|c||c|}\hline
  $P$ & $Q$ & $P\then Q$ \\\hline\hline
  $T$ & $T$ & $T$ \\\hline
  $T$ & $F$ & $F$ \\\hline
  $F$ & $T$ & $T$ \\\hline
  $F$ & $F$ & $T$ \\\hline
\end{tabular}\hfill

\vfill

  {\bf Outline for Direct Proof}

  \fbox{\parbox{2in}{
      {\bf Proposition} If $P$, then $Q$.\\
      {\it Proof.}  Suppose $P$.\\
      $\vdots$\\
      Therefore, $Q$.\QED
  }}
\end{center}

\end{frame}

\bfr{Example proof development}
\fbox{\parbox{\textwidth}{
    {\bf Proposition} If $x$ is odd, then $x^2$ is odd.

    {\sl Proof.} Suppose $x$ is odd.

    $\vdots$

    Therefore $x^2$ is odd, {\em for some reason.} \QED}}
\end{frame}


\bfr{Example proof development}
\fbox{\parbox{\textwidth}{
    {\bf Proposition} If $x$ is odd, then $x^2$ is odd.

    {\sl Proof.} Suppose $x$ is odd.

    Then $x=2a+1$ for some $a\in\ints$, by definition of an odd number.

    $\vdots$

    Therefore $x^2$ is odd, {\em for some reason.} \QED}}
\end{frame}

\bfr{Example proof development}
\fbox{\parbox{\textwidth}{
    {\bf Proposition} If $x$ is odd, then $x^2$ is odd.

    {\sl Proof.} Suppose $x$ is odd.

    Then $x=2a+1$ for some $a\in\ints$, by definition of an odd number.

    $\vdots$

    Thus $x^2 = 2b+1$ for an integer $b$, {\em for some reason.}

    Therefore $x^2$ is odd, by definition of an odd number. \QED}}
\end{frame}

\bfr{Example proof development}
\fbox{\parbox{\textwidth}{

    \mbox{\bf Proposition} If $x$ is odd, then $x^2$ is odd.

    \mbox{\it Proof.} Suppose $x$ is odd.

    Then $x=2a+1$ for some $a\in\ints$, by definition of an odd number.

    Thus
    \begin{align*}
      x^2 &= (2a+1)^2  & \mbox{by substitution}\\
      &= 4a^2 + 4a + 1 &\mbox{by algebra}\\
      &= 2(2a^2 + 2a) + 1 &\mbox{by algebra}
    \end{align*}
    If we let $b=2a^2+2a$ then $b$ is an integer, by math facts.

    Thus $x^2 = 2b+1$ for an integer $b$, by substitution.

    Therefore $x^2$ is odd, by definition of an odd number. \QED
    }}
\end{frame}


\bfr{Example proof development}

    {\bf Proposition} Let $a,b,c\in\ints$.  If $a\mid b$ and $b\mid
    c$, then $a\mid c$.

\end{frame}


\bfr{Example proof development}
\fbox{\parbox{\textwidth}{
    {\bf Proposition} Let $a,b,c\in\ints$.  If $a\mid b$ and $b\mid
    c$, then $a\mid c$.

    {\sl Proof.}

    $\vdots$

 \QED}}
\end{frame}

\bfr{Example proof development}
\fbox{\parbox{\textwidth}{
    {\bf Proposition} Let $a,b,c\in\ints$.  If $a\mid b$ and $b\mid
    c$, then $a\mid c$.

    {\sl Proof.} Suppose  $a\mid b$ and $b\mid c$.

    $\vdots$

    Therefore $a\mid c$ \QED}}
\end{frame}

\bfr{Example proof development}
\fbox{\parbox{\textwidth}{
    {\bf Proposition} Let $a,b,c\in\ints$.  If $a\mid b$ and $b\mid
    c$, then $a\mid c$.

    {\sl Proof.} Suppose  $a\mid b$ and $b\mid c$.

    By definition, $a\mid b$  means there exists $d\in\ints$ with 
    \begin{align*} b = ad \end{align*}
    By definition, $b\mid c$  means there exists $e\in\ints$ with 
    \begin{align*} c=be \end{align*}

    $\vdots$

    Thus $c= ax$ for some $x\in\ints$.

    Therefore $a\mid c$, by definition. \QED}}
\end{frame}
\bfr{Example proof development}
\fbox{\parbox{\textwidth}{
    {\bf Proposition} Let $a,b,c\in\ints$.  If $a\mid b$ and $b\mid
    c$, then $a\mid c$.

    {\sl Proof.} Suppose  $a\mid b$ and $b\mid c$.

    By definition, $a\mid b$  means there exists $d\in\ints$ with 
    \begin{align*} b = ad \end{align*}
    By definition, $b\mid c$  means there exists $e\in\ints$ with 
    \begin{align*} c=be \end{align*}

    By combining equations these two equations, we get
    \[ c = be = (ad)e = a(de) \]

    Let $x=de$, then $x\in\ints$.

    Thus $c= ax$ for some $x\in\ints$.

    Therefore $a\mid c$, by definition. \QED}}
\end{frame}

\bfr{Proof by cases}

{\bf Proposition} If $n\in\ints$ then $1+(-1)^n(2n-1)$ is a multiple
of 4.

Proof.  Suppose $n\in\ints$.

Then $n$ is either even or odd.

{\bf Case 1.}  Suppose $n$ is even.  Then $n=2k$ for some $k\in\ints$
and $(-1)^n = 1$.  Thus \[
1 +(-1)^n(2n-1) = 1 + (1)(2\cdot 2k -1) = 4k
\]
which is a multiple of 4.

{\bf Case 2.}  Suppose $n$ is odd.  Then $n=2k+1$ for some $k\in\ints$
and $(-1)^n = -1$.  Thus \[
1 +(-1)^n(2n-1) = 1 - (2(2k+1) -1) = -4k
\]
which is a multiple of 4.

These cases show that $1+(-1)^n(2n-1)$ is always a multiple of 4.

\end{frame}

\bfr{Without loss of generality}

{\bf Proposition}  If two integers have opposite parity, then their
sum is odd.

{\sl Proof.}  Suppose $m$ and $n$ are integers with opposite parity.

Without loss of generality, suppose $m$ is even and $n$ is odd.

Thus $m=2a$ and $n=2b+1$ for some $a,b\in\ints$.

Therefore $m+n=2a+2b+1 = 2(a+b)+1$, which is odd by definition.
\end{frame}


\bfr{Contrapositive Proof}
\newcommand{\row}[6]{#1&#2&#3&#4&#5&#6\\\hline}
\begin{center}
  If $P$, then $Q$.

  If $\neg Q$, then $\neg P$.

  \vfill

  \begin{tabular}{|c|c||c|c||c|c|c|}\hline
    $P$ & $Q$ & $\neg Q$ & $\neg P$ & $P\then Q$ & $\neg Q\then\neg
    P$\\\hline\hline
    \row TTFFTT
    \row TFTFFF
    \row FTFTTT
    \row FFTTTT
  \end{tabular}

  \vfill
  
$(P\then Q) \iff (\neg Q \then \neg P)$


\end{center}

\end{frame}

\bfr{Contrapositive Proof}
\begin{center}

  {\bf Outline for Contrapositive Proof}

  \fbox{\parbox{2in}{
      {\bf Proposition} If $P$, then $Q$.\\
      {\it Proof.}  Suppose $\neg Q$.\\
      $\vdots$\\
      Therefore, $\neg P$.\QED
  }}
\end{center}
\end{frame}

\bfr{Example Contrapositive Proof}

{\bf Proposition} Suppose $x\in\ints$.  If $7x+9$ is even, then $x$ is
odd.

{\sl Proof.}  (Contrapositive)

Suppose $x$ is not odd.

Thus $x$ is even, and $x=2a$ for some $a\in\ints$.

Then $7x+9 = 7(2a) + 9 = 14a + 8 + 1 = 2(7a+4) + 1$.

Thus $7x+9 = 2b+1$ where $b$ is the integer $7a+1$.

By definition, $7x+9$ is odd.

Therefore $7x+9$ is not even.

\end{frame}

\bfr{Congruence of Integers}

{\bf Definition 5.1}  Given $a,b\in\ints$ and $n\in\nats$, we say that
$a$ and $b$ are {\bf congruent modulo $n$} if $n\mid(a-b)$.  We
express this as $\equivmod{a}{b}{n}$.

\newcommand{\equivexamp}[3]{
\item $#1 \equiv #2\ (\mbox{mod } #3)$ because $#3 \mid (#1 - #2)$.
  }
\newcommand{\nequivexamp}[3]{
\item $#1 \not\equiv #2\ (\mbox{mod } #3)$ because $#3 \nmid (#1 - #2)$.
  }
{\bf Example 5.1}
\begin{enumerate}
\equivexamp {9}{1}{4}
\equivexamp {6}{10}{4}
\nequivexamp {14}{8}{4}
\equivexamp {20}{4}{8}
\equivexamp {17}{(-4)}{3}

\end{enumerate}

\vfill

{\sl They have the same remainder upon division by $n$.}
\end{frame}

\bfr{Proof of Congruence}

{\bf Proposition}  If $\equivmod{a}{b}{n}$ then $\equivmod{a^2}{b^2}{n}$.

{\it Proof.}  Suppose $\equivmod{a}{b}{n}$.

By definition, $n\mid (a-b)$.

By definition, $a-b = nc$ for some $c\in\ints$.
\begin{align*}
  a-b &= nc\\
  (a-b)(a+b) &= nc(a+b)\\
  a^2 - b^2 &= nc(a+b)
\end{align*}
Since $c(a+b)\in\ints$, this tells us that $n\mid (a^2-b^2)$.

By definition, $\equivmod{a^2}{b^2}{n}$.

\end{frame}


\bfr{Proof by Contradiction}
\newcommand{\rowfive}[5]{#1&#2&#3&#4&#5\\\hline}
\begin{center}
  \begin{tabular}{|c|c||c|c||c|}\hline
    $P$ & $C$ & $\neg P$ & $C\land \neg C$ & $(\neg P) \then
    (C\land\neg C)$\\\hline\hline
    \rowfive TTFFT
    \rowfive TFFFT
    \rowfive FTTFF
    \rowfive FFTFF
  \end{tabular}

  \vfill

  {\bf Outline for Proof by Contradiction}

  \fbox{\parbox{2in}{
      {\bf Proposition} $P$.\\
      {\it Proof.}  Suppose $\neg P$.\\
      $\vdots$\\
      Therefore, $C \land \neg C$.\QED
  }}
\end{center}
\end{frame}

\bfr{Example Proof by Contradiction}

{\bf Proposition}  If $a,b\in\ints$, then $a^2-4b\neq 2$.

{\it Proof.}  Suppose this is false.
\pause

There exist $a,b\in\ints$ with $a^2-4b=2$.
\pause

Then $a^2 = 4b + 2 = 2(2b+1)$, is even.\\
\pause
So $a$ is even, so $a=2c$ for
some $c\in\ints$.
\pause
\begin{align*}
  (2c)^2 - 4b &= 2\\
  4c^2 - 4b &= 2\\
  2c^2 - 2b &= 1\\
  2(c^2 - b) &= 1
\end{align*}
\pause
But $c^2-b\in\ints$.
\pause

Which implies that 1 is even, which is a
contradiction. \QED

\end{frame}



\bfr{Compare Contrapositive with Contradiction}
\begin{center}
  {\bf Outline for Contrapositive Proof}

  \fbox{\parbox{2in}{
      {\bf Proposition} If $P$, then $Q$.\\
      {\it Proof.}  Suppose $\neg Q$.\\
      $\vdots$\\
      Therefore, $\neg P$.\QED
  }}

  \vfill

  {\bf Outline for Proof by Contradiction}
  \fbox{\parbox{2in}{
      {\bf Proposition} $P$.\\
      {\it Proof.}  Suppose $\neg P$.\\
      $\vdots$\\
      Therefore, $C \land \neg C$.\QED
  }}
\end{center}

\end{frame}

\bfr{Proving a Conditional Statement with Contradiction}

\begin{center}
  
  \fbox{\parbox{2in}{
      {\bf Proposition} If $P$, then $Q$.\\
      {\it Proof.}  Suppose $P$ and $\neg Q$.\\
      $\vdots$\\
      Therefore, $C \land \neg C$.\QED
  }}
\end{center}
\end{frame}

\bfr{Example}

{\bf Proposition}  Suppose $a\in\ints$.  If $a^2$ is even, then $a$ is
even.

{\it Proof.}  Suppose $a^2$ is even and $a$ is not even.

Then $a$ is odd.

Then $a=2c+1$ for some $c\in\ints$.

Then \begin{align*}
  a^2 &= (2c+1)^2 \\
  &= 4c^2 + 4c + 1\\
  &= 2(2c^2 + 2c) + 1 
\end{align*}

Then $a^2$ is odd.

Thus $a^2$ is even and odd, which is a contradiciton. \QED

\end{frame}


\bfr{Example}

{\bf Proposition}  If $a,b\in\ints$ and $a\geq 2$, then $a\nmid b$ or
$a\nmid (b+1)$.

{\it Proof.}  Suppose $a,b\in\ints$ with $a\geq 2$ and it is not true
that $a\nmid b$ or
$a\nmid (b+1)$.

Then $a\mid b$ and $a\mid (b+1)$.

Then there are $c,d\in\ints$ with $b=ac$ and $b+1=ac$.

Subtracting the equations gives
\begin{align*}
  1 &= ad-ac\\
  &= a(d-c)
\end{align*}
Since $a$ is positive, $d-c$ is positive.
So
\begin{align*}
  a &= 1/(d-c) < 2
\end{align*}
Therefore $a\geq 2$ and $a<2$, a contradiction. \QED

\end{frame}

\bfr{Example}

{\bf Proposition}  Every non-zero rational number can be expressed as
a producto of two irrational numbers.

{\it Proof.}  Reword the proposition:
If $r$ is a non-zero rational number, then $r$ is the product of two
irrational numbers.

Suppose $r$ is a non-zero rational number.

Then $r=a/b$ for $a,b\in\ints$.  Also,
$r = \sqrt{2}({r}/{\sqrt{2}})$.

We know $\sqrt{2}$ is irrational, so we need to prove that
$r/{\sqrt{2}}$

To show this, assume $r/\sqrt{2}$ is rational.  Then
${r}/{\sqrt{2}} = {c}/{d}$
for some $c,d\in\ints$.

So
\[
  \sqrt{2} = r\frac{d}{c}
  = \frac{a}{b}\frac{d}{c}
  = \frac{ad}{bc}
\]
Which means $\sqrt{2}$ is rational, which is a contradiction.

Therefore $r/\sqrt{2}$ is irrational.

Therefore $r=\sqrt{2}\cdot r/\sqrt{2}$ is a product of two irrational
numbers. \QED

\end{frame}

\end{document}
