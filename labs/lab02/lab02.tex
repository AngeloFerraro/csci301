\documentclass{article}
\pagestyle{empty}
\usepackage[margin=1in]{geometry}
\usepackage{fancyvrb}
\usepackage{multicol}
\title{CSCI 301, Lab \# 2}
\author{Winter 2018}
\date{}
\begin{document}

\maketitle
%\setlength{\columnsep}{2em}
\begin{multicols}{2}


\paragraph{Goal:} The purpose of this lab is write some  code in 
  which functions are both passed as parameters, and returned as values. 

\paragraph{Due:} Your program, named {\tt lab02.rkt}, must be submitted to
  Canvas before midnight, Monday, Jan 29.

  \paragraph{Unit tests:}
  At a minimum, your program must pass the unit tests found in the
  file {\tt lab02-test.rkt}.  Place this file in the same folder
  as your program, and run it;  there should be no output.

\paragraph{Program:} Write a {\sc Scheme} procedure called {\tt I} that
  does approximate numeric integration, as a companion to the {\tt D}
  procedure from the lecture notes:
  \begin{Verbatim}[frame=single]
(define D
  (lambda (f)
    (let* ((delta 0.00001)
           (two-delta (* 2 delta)))
      (lambda (x)
         (/ (- (f (+ x delta))
               (f (- x delta)))
           two-delta)))))
  \end{Verbatim}
  This was found using the numeric approximation to the derivative:
  \[
  \frac{d}{dx} f(x) \approx \frac{f(x+\Delta x) - f(x)}{\Delta x}
  \]
  This can be made more accurate simply by making $\Delta x$ smaller,
  so long as we don't get too small for our computer representation.
  
  We have a numeric approximation for the {\em definite} integral, as
  follows:
\begin{eqnarray}
  \int_a^b f dx  &=&  \sum _{i=0}^{n} f(a+i\Delta x) \label{summation}
\end{eqnarray}
where $a+n\Delta x = b$.  This is the ``add up the skinny rectangles''
definition of the definite integral.

However, we don't have a simple definition of the {\em indefinite}
integral, sometimes called the antiderivative, like we do for the
derivative.  There's no way to give a simple numerical definition, for
example, for $I$ in the following equation:
\begin{eqnarray}
\int f(x) dx = I(f)
\end{eqnarray}
The indefinite integral of a function,
however, can be defined as a function which, when plugged into a {\em
  definite} integral, with limits $a$ and $b$, will return a value.
In fact, it's quite simple:
\begin{eqnarray}
\int_a^b f dx  =  I(b) - I(a)
\end{eqnarray}

This gives us our key.  We will define a Scheme function {\tt i}, for
integrate, which, when given a function as an argument, {\tt (i f)},
returns a function of two parameters, {\tt a} and {\tt b}.  When
{\em this returned function} is given two numbers, it computes the definite
integral of {\tt f} from {\tt a} to {\tt b}, using the summation of
skinny rectangles technique.

Here are some examples of how it should work.  Since
\begin{eqnarray}
  \int_2^4 2x\ dx = 4^2 - 2^2 = 12
\end{eqnarray}
then, approximately,
\begin{Verbatim}
  ((i (lambda (x) (* 2 x))) 2 4)
      => 12
  ((i (d (lambda (x) (* x x)))) 2 4)
      => 12
\end{Verbatim}

Note that we can't give a tolerance for the error for our integral
function, since we don't really know how big the error will be when we
add up the skinny rectangles.  Further, greater accuracy can only be
achieved with more terms in the summation, which (unlike the
derivative) will make our integral slower and slower.  For this
assignment, we'll make a decent compromise and let the number of
rectangles be fixed at $n=100,000$, which should give us pretty good
accuracy without taking too long.

\end{multicols}

\end{document}

\end
